\documentclass{./assets/wfis}
\usepackage[utf8]{inputenc}
\usepackage{amsmath}
\usepackage{tabularx}
\usepackage[hidelinks]{hyperref}
\usepackage{afterpage}
\usepackage{biblatex} 
\addbibresource{assets/references.bib}

\usepackage{framed}
\usepackage{listings}
\usepackage[most]{tcolorbox}
\usepackage{hyphenat}

\begin{document}

\tytul{Wykrywanie zmian stanu skupienia przy użyciu aparatu EEG}
\autor{Mateusz Kojro}
\nralbumu{389105}
\promotor{dr hab. Krzysztofa Wardy}
\katedra{Teorii Ciała Stałego}
\kierunek{informatyka}
\specjalnosc{informatyka stosowana}
\typpracy{inżynierska}
\specjalizacja{Algorytmy i Programowanie}


\stronatytulowa

\chapter{Wstęp}
\section{Other Work}

\chapter{Metody}
\section{EEG}
Elektroencefalografia (EEG) to badanie polegające na pomiarze względnych potencjałów elektrycznych na czaszce pacjenta za pomocą różnego rodzaju elektrod. Zebrane w ten sposób informacje pozwalają na określenie względnej aktywności różnych obszarów mózgu w porównaniu do wartości obserwowanych u zdrowej osoby lub w relacji na bodziec (np. zamknięcie oczu).

Elektroencefalografia tradycyjnie wykorzystywana jest między innymi do diagnozowania schorzeń neurologicznych takich jak epilepsja, nowotwory mózgu czy zaburzenia snu. Podczas badania, operator wizualnie analizuje wykresy sygnałów o różnych częstotliwościach w czasie. W związku z relatywnie niskim poziomem sygnału do szumu oraz skomplikowanymi relacjami ukrytymi w wynikach oraz łatwy dostęp do aparatury pomiarowej pozwalający na stosunkowo proste zbieranie znaczących ilości danych rozpoczęto prace nad zastosowaniem algorytmów uczenia maszynowego w celu diagnozy innych jednostek chorobowych takich jak ADHD, dysleksja czy schizofrenia \cite{ahire_comprehensive_2022, joshi_review_2021, clarke_eeg_2002}.

\subsection{Mechanizm Działania}
% Potencjał czynnościowy (ang. \textit{action potential}) zmienia wewnętrzne napięcie neuronu o $\approx100mV$ na czas około $3ms$, biorąc pod uwagę odległość elektrod od źródła sygnału (natężenie pola elektrycznego spada z kwadratem odległości) oraz właściwości tłumiące tkanek ludzkich, pomiar aktywności pojedynczych neuronów za pomocą elektrod umieszczonych na skórze głowy jest niemożliwe.

Układ nerwowy człowieka przekazuje oraz przetwarza informacje za pomocą impulsów elektrycznych (potencjałów czynnościowych) wytwarzanych w komórkach nerwowych (neuronach), z pomocą mechanizmu xxxxx. Z względu na dużą liczbę neuronów w ludzkim mózgu ($\approx10^9$), ich małe rozmiary (xxxx) małe napięcie impulsu potencjału czynnościowego ($\approx100mV$) oraz krótki czas jego trwania ($\approx2ms$) niemożliwe jest badanie pracy pojedynczych neuronów, a jedynie średnich wartości milionów zsynchronizowanych impulsów na przestrzeni centymetrów (xxxxxx). Ograniczenia te powodują że różne obszary mózgu mogą być badane z różną dokładnością (najlepsze sprawność otrzymuje się w rejonach mózgu znajdujących się blisko powierzchni skóry zawierających dużą liczbę tzw. neuronów piramidowych – kora przedczołowa, część mózgu odpowiedzialna za xxxxx)

\subsection{Metodyka Badania}

\section{Sprzęt}

\chapter{Wyniki}
\section{Demografia}

\printbibliography

\end{document}